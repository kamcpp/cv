\documentclass[10pt,a4paper]{article}
\usepackage{lipsum}
\usepackage{fancyhdr}
\usepackage[margin=1.1 in]{geometry}
\usepackage{layout}
\usepackage{sectsty}

\pagestyle{fancy}
\lhead{\textbf{Kamran Amini} \\ Mobile: (+98) 935 45 92 265 \\ Email: kam.cpp@gmail.com}
\rhead{\parbox[b][5mm][t]{0.5\textwidth}{\raggedleft Page \thepage}}
\renewcommand{\headrulewidth}{0.4pt}
\renewcommand{\footrulewidth}{0.4pt}

\setcounter{secnumdepth}{0}

\sectionfont{\fontsize{12}{15}\selectfont}
\setlength{\headsep}{30pt}

\begin{document}

\thispagestyle{fancy}

\section{IN SUMMARY}
As a highly active, motivated and 'computer science loyal' professional software architect with more than of 10 years of experience in development, analysis, design and architecting large scale, national and international projects, my skills are not limited to a specific development platform and my hybrid distributed multi-tier architectural solutions vast from back-end services written in C/C++, Java or Python to front-end applications written in C/C++, Java or even deployed on Web, with consideration of critical architectural factors like Distribution of Data and Computation across multiple domains and clusters, Fault Tolerance of services, High Availability of services, System's Single Points of Failure, overall Performance and Efficiency of the system, Security factors and hardening these factors across the system's domains, etc. As an architect of national projects, I've always faced strict and risky conditions which Security, Data Consistency and Data Correctness have been always very important factors. As an example, architecture of Distributed ePassport Issuance System (Kawthar System) for Iran's Ministry of Foreign Affairs, was led by me and it is operating world wide in Iran's Consular Offices and Embassies in more than of 70 countries.

\rule {14cm}{0.4pt}

\section{EMPLOYMENT EXPERIENCES}

\subsubsection{\textnormal {April 2015 - Present} \\ \textnormal {Noavaran Tejarat Gostar Naeem} \\ Chief Technical Officer \\ Senior Software Architect \\ Senior C/C++/Java SE Developer}
  \setlength{\leftskip}{0.5cm}
  \setlength{\rightskip}{1cm}
  N.T.Naeem company operates in many fields related to Identity Solutions, RFID equipped systems and Security. As a CTO, I was responsible for every technical decision and management of technical teams, both hardware and software teams. As an architect, I was responsible for architecting projects ranging from middle to large scale systems. Some highlights about my skills and roles are mentioned below.
  \begin{itemize}
    \setlength{\rightskip}{2cm}
    \setlength\itemsep{0em}
    \item \small I standardized technical procedures with use of Debian-based Linux distros alongside with Git, Jenkins, Redmine Issue Tracker, etc.
    \item \small Managing more than 15 software and hardware developers and architects organized in 6 different parallel teams, working on various research and development topics.
    \item \small Strategic decision making for future of technical earnings and areas of study.
    \item \small Architecting RFID equipped systems. Coaching and development alongside with embedded programmers for working 
    \item \small Architecting systems using on-edge technologies and concepts with cooperation of other architects.
    \item \small Ability to merge and organize different parts of a heterogeneous system using different toolsets and programming languages and making SOA dominant in architectural decisions.
    \item \small Using C/C++, Java SE, Java EE and Python whenever a specific feature or factor acceptance were needed.
    \item \small Managing and cooperation with research teams operating in Image Processing, Machine Learning, NLP, Security, Hardware, Cryptography, Mathematics and other computer science related fields.
    \item \small Managing, development and leading the architectural teams in various international and national projects like Kawthar  Geographically Distributed ePassport Issuance System, Network Analysis Engine, Data Science Analysis Framework, Consular Portal for Iran's Ministry of Foreign Affairs, Distributed Crawler Engine, ABIS Engines, Biometric Data Enrolment Systems, etc.
    \item \small Design and implementation of a Network Analysis distributed engine purely written in C. Supervising all data structures and algorithms proposed by research team and coaching developers and architects through implementation phase.
    \item \small Supervision on grammar design and implementation of a query language using Yacc/Lex for querying Network Analysis engine.
    \item \small Working on open source projects like Hottentot RPC Framework, C/C++ Corelibs, Avicenna, etc.
    \item \small Cooperation and holding sessions with academic people and using the top most national knowledge to supply the research teams.
    \item \small Development and leading distributed systems architectures using various systems like Apache Hadoop Family, Apache Cassandra, Redis, Apache ZooKeeper
    \item \small Architecting scalable computation engines both vertically and horizontally covering FT, HA, Replication, Consistency  concepts.
    \item \small Architecting and development of RPC related solutions for ensuring security and performance in all distributed systems.
     \item \small BigData Solutions
     \item \small Public Key Infrastructure
     \item \small Security Tokens
     \item \small Embedded Programming (Raspberry Pi, 
  \end{itemize}
  \setlength{\leftskip}{0pt}
  \setlength{\rightskip}{0cm}
	  
\subsubsection{\textnormal {February 2015 - June 2015} \\ \textnormal {Sahab Pardaz Co.} \\ Senior Software Architect \\ Distributed Systems Architect \\ Senior Java SE/EE Developer \\ C++ Developer}
	\setlength{\leftskip}{0.5cm}
  \setlength{\rightskip}{1cm}
  Sahab Pardaz Co. is a company making scalable and distributed enterprise level applications using C/C++ and Java based solutions. Some highlights about my skills and roles are mentioned below.
  \begin{itemize}
    \setlength{\rightskip}{2cm}
    \setlength\itemsep{0em}
    \item \small Dominant concepts, technologies and tools were Ubuntu, C/C++, Java SE 7 and Java EE 7, Git, Mercury, Redmine Issue Tracker, Jenkins and Test Driven Development.
    \item \small Mainly worked as a Senior Distributed System Architect, focusing on scalable systems with high throughputs and balancing the load along many clusters.
    \item \small Participation in sessions with other architectures to analyse and design different parts of a system.
    \item \small Supervising C/C++ and Java developers
    \item \small Designing the tests which they would be assigned to developers.
    \item \small Maven-based versioning system for writing embedded stubs for facilitate test scenario.
    \item \small Apache Hadoop Ecosystem, Apache Cassandra, Apache Ignite, Apache ZooKeeper, Apache Storm and Spark were among of those many open source Java based solutions which they were analysed and used for making scalable distributed systems.
    \item \small RPC solutions were GRPC (Google RPC), ZeroIce.
  \end{itemize}
  \setlength{\leftskip}{0pt}
  \setlength{\rightskip}{0cm}
	  
\subsubsection{\textnormal {March 2013 - June 2015} \\ \textnormal {Mojtama E Fanni Tehran (Tehran Technical Complex)} \\ Director of Programming Department \\ C++/Java SE/EE Instructor \\ Senior Java EE Developer/Architect}
  \setlength{\leftskip}{0.5cm}
  \setlength{\rightskip}{1cm}
  \begin{itemize}
    \setlength{\rightskip}{2cm}
    \setlength\itemsep{0em}
    \item \small Director of Programming department
    \item \small Working on an educational system called EduSys as architect and senior developer
    \item \small SOA using Spring WS and Microsoft WCF
    \item \small Web tier using Spring MVC and Spring Security frameworks
    \item \small PostgreSQL were used as database management system
    \item \small Ubuntu Server distribution were used as our server environment.
  \end{itemize}
  \setlength{\leftskip}{0pt}
  \setlength{\rightskip}{0cm}
	  
\subsubsection{\textnormal {December 2011 - June 2013} \\ \textnormal {Sharif CERT Center} \\ Java SE/EE Developer \\ C/C++ Developer/Architect}
  \setlength{\leftskip}{0.5cm}
  \setlength{\rightskip}{1cm}
	\begin{itemize}
		\setlength{\rightskip}{2cm}
    \setlength\itemsep{0em}
	  \item \small Network security related applications written using Java, C/C++ under Linux environments
		\item \small Using SVN and Git alongside Eclipse IDE for Java development
		\item \small Using Test Driven Development technique for application development
		\item \small Research and develop on Network Security concerns (L2 and L3)
		\item \small Working with algorithms and data structures which they are involved in security related topics.
		\item \small Malware analysis, Traffic analysis, Attack analysis
		\item \small 2012 May, Member of Technical Team, CTF Competitions, Iran's Second National Hacking Contest, Sharif University of Technology. Designer and Auditor for Java Secure Coding Question
		\item \small 2013 Feb, Member of Technical Team, CTF Competitions, Iran's Second National Hacking Contest, Sharif University of Technology. Designer and Auditor for C++ Secure Coding Question
	\end{itemize}
  \setlength{\leftskip}{0pt}
  \setlength{\rightskip}{0cm}
	  
\subsubsection{\textnormal {October 2011 - June 2012} \\ \textnormal {Sharif Engineering Process Development Co.} \\ C++/Qt Developer}
  \setlength{\leftskip}{0.5cm}
  \setlength{\rightskip}{1cm}
  \begin{itemize}
		\item \small \textit{SEPDCo Gas Network Simulator System} \\
     Working on a gas network and pipeline designer and simulator application written entirely in C++ using Qt framework. I was developer for the desktop application which was written in C++ and Qt View Framework. I used Qt View Framework classes like QWidget, QMainWindow, QGraphicsScene, QGraphicsView alongside signals and slots with all other needed concepts in Qt. The application had no database and everything was serialized using XML or Binary formatters and written to a file. We were using Microsoft Visual Studio 2008 for developing the project.
  \end{itemize}
  \setlength{\leftskip}{0pt}
  \setlength{\rightskip}{0cm}
  
\subsubsection{\textnormal {September 2009 - August 2011} \\ \textnormal {Septa Co.} \\ C\#.NET Developer/Architect}
  \setlength{\leftskip}{0.5cm}
  \setlength{\rightskip}{1cm}
  We used Microsoft Visual Studio 2008 as our primary development IDE and Microsoft Visual Source Safe 2005 for our source version control system. Project types were varied from Windows From applications to ASP.NET Web Forms and WCF Services.
  \begin{itemize}
		\item \small \textit{SeptaNSF; A .NET based Framework} \\
			This was a .NET based framework implemented for internal usages and later became a product. It also had a code generator which was able to generate Business Logic Layer entities and their adapters. It was entirely written in C\#.NET 3.5 and I was one of its developers and designers. I was developer and designer for classes related to WCF services, classes related to Inter-Process Communications like Sockets, Pipes and classes related to mathematical concepts related to cryptography. \\
		\item \small \textit{\texttt{www.1st.ir} B2B Portal} \\
			It was one of the first Iranian Business 2 Business portals. Written in ASP.NET C\#.NET 3.5, using a Microsoft SQL Server 2008 R2 database. All parts relating to WCF web services were developed by me. Also, portal integration with other products using web services were mine. \\
		\item \small \textit{PayamGostar Application} \\
			A business directory desktop application written entirely in C\#.NET 3.5 and it was using a Microsoft SQL Server 2008 R2 database. It allowed people to search, save and edit their customers and it had a bank of businesses information called business directory. It was integrated with 1st portal (previous section) and bought application instances had to be activated via portal. It was checking user's activation status using WCF web services over a HTTPS channel. In addition, new information for businesses could be synchronized with \texttt{www.1st.ir} portal using WCF web services. \\
		\item \small \textit{Septa Update Center} \\
			An update/patch system integrated with development procedures to provide updates for company's products. It was entirely written in C\#.NET 3.5. An updater service with a desktop updater application were given to clients for updating their product instances. It was fully designed and implemented by me and now it is working as the main update/patch package for Septa Co. products. It is implemented using WCF technology for downloading patches, updates, checking activation and subscription status and deactivating product instances. \\
		\item \small \textit{Septa Automation System} \\
			It was an internal ASP.NET C\#.NET 3.5 website for entrance system, salary calculations, product instances, serial code generation, tracking sold products, etc. It was all implemented by me. It was using ASP.NET membership bundle as its authentication system.
  \end{itemize}
  \setlength{\leftskip}{0pt}
  \setlength{\rightskip}{0cm}

\subsubsection{\textnormal {2009 June-2009 September} \\ \textnormal {Shabakeh Pardaz Rayaneh Co.} \\ C++ Developer/Architect, C\#.NET Developer}
\subsection{}
  \setlength{\leftskip}{0.5cm}
  \setlength{\rightskip}{1cm}
	\begin{itemize}
		\item \small \textit{Isfahan Province Electricity Distribution Center, 121 Call Center} \\
		This was the same project at previous company, Aramin IT, which was transferred to Shabakeh Pardaz Rayaneh Co. and I moved to new company to continue to work on the project. \\
		\item \small \textit{Recording and Archiving System at Isfahan Central Jails Organization} \\
		The archiving system was implemented using C\#.NET and Microsoft SQL Server 2005. Many database related issues like heavy loads and load balancing, partitioning, indexes, etc were encountered and studied.
	\end{itemize}
	\setlength{\leftskip}{0pt}
  \setlength{\rightskip}{0cm}
		
\subsubsection{\textnormal {2008 March -2009 June} \\ \textnormal {Aramin IT Co.} \\ C++ Developer/Architect, C\#.NET Developer}
  \setlength{\leftskip}{0.5cm}
  \setlength{\rightskip}{1cm}
  \begin{itemize}
    \item \small \textit{Isfahan Province Electricity Distribution Center, 121 Call Center} \\					
		This was a set of projects using Visual C++ and C\#.NET which they were responsible for handling calls at Isfahan Electricity Distribution Center. There was a service written in C++ which it had to monitor a Panasonic TD 200 call center. The service was using TAPI API for controlling the call center. Also, there were some desktop applications written in C\#.NET which they allowed human agents and managers to monitor, accept, reject and redirect the calls. Database system used for this project was Micrsoft SQL Server 2005.
	\end{itemize}
  \setlength{\leftskip}{0pt}
  \setlength{\rightskip}{0cm}

\rule {14cm}{0.4pt}

\section{EDUCATION}

2004-2008 \\
University of Isfahan \\
Isfahan, Iran \\
\textbf{Bachelor of Software Engineering}

\rule {14cm}{0.4pt}

\section{SCIENTIFIC AND PROFESSIONAL SKILLS}

\begin{itemize}
  \item \small \textit {Computer Science} \\
  Data Structures, Algorithm Design, Automata Theory, Turing Machine and Computation Theory, Linear and Abstract Algebra, Information Theory, Cryptography, Quantum Cryptography and Key Distribution, Quantum Computation, Artificial Intelligence and Machine Learning,  Relational Databases, Graph Databases, Key-Value Databases, File Systems, Operating Systems, Kernel Architecture, Logical Circuits, Digital Circuits, Analogue Circuits, etc
  \item \small \textit {Programming Languages} \\
  C, Embedded C (ARM, AVR), C++, Java, Python
  \item \small \textit {Databases} \\
  PostgreSQL, Oracle RDBMS, MySQL, MongoDB, Apache Cassandra, Neo4j, Redis, etc
  \item \small \textit {Operating Systems} \\
  Linux Distros (Ubuntu, Debian, CentOS), FreeBSD
  \item \small \textit {Source Version Controls} \\
  Git, SVN, Mercurial
  \item \small \textit {Software Engineering} \\
  MVP, Agile methodologies (Scrum), RUP, etc
\end{itemize}

\rule {14cm}{0.4pt}

\section{HOBBIES}

\end{document}