\documentclass[12pt,a4paper]{article}
\usepackage{amsmath}
\usepackage{graphicx}
\usepackage[margin=1.0 in]{geometry}
\usepackage{longtable}
\usepackage{color}
\usepackage{hyperref}

\hypersetup{
    colorlinks=true,
    linktoc=all,
    linkcolor=blue,
}

\newif\ifanswers
\answerstrue

\setlength{\LTleft}{-0.5in}

%\setcounter{secnumdepth}{-1} 

\begin{document}
\title{Curriculum Vitae}
\author{Kamran Amini}
\maketitle

\tableofcontents
\newpage

\section{Overall Information}

	\subsection{Personal Information}
		\textbf{Name}: Kamran \\
		\textbf{Family}: Amini \\
		\textbf{Birth date}: 1987 Apr 25 \\
		\textbf{Birth place}: Tehran, Tehran, Iran \\
		\textbf{Nationality}: Iranian \\
		\textbf{Current living place}: Tehran, Tehran, Iran 
			
	\subsection{Contact Information} 
		\textbf{Phone}: \indent \texttt{+98 21-33436149} \\
		\textbf{Mobile}: \indent \texttt{+98 935-45-92-365} \\
		\textbf{Work email}: \indent \texttt{kam.cpp@gmail.com} \\ \indent \indent \indent \indent \indent \texttt{k.amini@mftvanak.com} \\ 
		\textbf{GMail}: \indent \texttt{kam.cpp@gmail.com} \\
    	\textbf{Home Address}: \indent \texttt{No.12, Mahzoon Alley, Tayeb St., Khorasan Sq., Tehran, Tehran, Iran} \\
    	\textbf{Sourceforge Address}: \indent \texttt{https://sourceforge.net/u/kamranamini/profile/}
    	\textbf{GitHub Address}: \indent \texttt{https://github.com/kamcpp} \\
			
	\subsection{Last Employment} 
		\textbf{Organization}: Mojtama E Fanni E Tehran (Tehran Institute of Technology), Vanak delegacy \\
		\textbf{Webstie}: \texttt{http://www.mftvanak.com} \\
		\textbf{Roles}: \\ \indent Director of Web and Development Department \\ \indent Software Architect, Software Developer \\ \indent Instructor and Teacher \\
		\textbf{Since}: 2011 December \\
		\textbf{Working experience}: About 7 years \\
		\textbf{Recent worked technologies}: Java SE, Java EE Specs and Implementations, Spring MVC, Spring Security, Spring WS, PostgreSQL, Ubuntu Linux, Eclipse, IntelliJ IDEA, Shell Scripting, C/C++, Microsoft Visual C\#.NET 2012 (.NET Framework 4.5)
	
	\subsection{Desired Employment as Software Architect or Developer}
	   \textbf{Desired working fields}: Network Programming, Security Implementation and Programming, Cryptography, Algorithms, Studying and implementation mathematical aspects of algorithms and cryptography algorithms, Network traffic analysis, Information analysis, Kernel Module Development, Unix and Linux Kernel Development, Driver Development \\
	   \textbf{Desired technologies}: C/C++ and Java in Linux environments, Kernel Development using C \\
	   \textbf{Roles}: Software architect, Software developer, Software designer, Instructor \\
	   \textbf{Desired salary}: Hourly and negotiable \\
	   \textbf{Type of cooperation}: Hourly \\
	   \textbf{Days of week}: Maximum 3 days of week \\
	   \textbf{Monthly working hours}: Negotiable \\
	   \textbf{Start date}: As soon as possible \\
	   \textbf{Insurance}: Yes
	   
	\subsection{Educational Information}
		\begin{itemize}
		\item High school Diploma in Mathematical Sciences, Sanie-far High school, Finished at 2005 with GPA 19/20
		\item Studied Bachelor in Software Engineering, University of Isfahan, started at 2005 September and left at 2008 September with 99 completed course units out of 141 course units and GPA 14.28/20
		\end{itemize}
			
	\subsection{Speaking languages} 
		Persian : Native \\
		English : Fluent \\
		French : Intermediate in Reading and Listening, Beginner in Writing and Speaking \\
		Japanese : Lower Intermediate, Studied for JLPT N1 but ceased now \\
		German : Beginner 
			 
	\subsection{Hobbies and Interests} 
		\textbf{Mathematics} Linear and Abstract Algebra, Probability Theory, Information Theory \\
		\textbf{Philosophy} Western Philosophy, Socilogy, Economics, Politics and Humanities \\
	    \textbf{Physics} Quantum Physics, Quantum Mechanics, Particle Physics, Relativity, Theory of Everything, Photonics \\
	    \textbf{Cryptography} Information Security, Network Security, Security Algorithms, Quantum Information Theory, Quantum Key Distribution, Quantum Cryptography \\
	    \textbf{Telecommunications} Modulation, Multiplexing, Codes, Telecommunication circuits, Photonics and optical fibers, IEEE 802.3 Protocols, Information Theory, Coding Theory, Computer Networks, etc \\ 
		\textbf{Music and musicology} Classic, Rock, Metal, etc
	 
\section{Scientific Career}
	\subsection{Interests}
		\begin{itemize}
			\item Algorithms, Information Theory, Cryptography, Quantum Information Theory, Quantum Cryptography
		\end{itemize}
	\subsection{Memberships}
		\begin{itemize}
			\item 2012 May, Member of Technical Team, CTF Competitions, Iran's Second National Hacking Contest, Sharif University of Technology. Designer and Auditor of Java Secure Coding Question
			\item 2013 Feb, Member of Technical Team, CTF Competitions, Iran's Third National Hacking Contest, Sharif University of Technology. Designer and Auditor of C++ Secure Coding Question
		\end{itemize}
	\subsection{Co-author}
		\begin{itemize}
			\item 2007 Co-author for Master Thesis "Pattern Recognition using Feature Extraction for R6 robots.", Islamic Azad University, Science and Research
			\item 2010 Co-author for Master Thesis "Author Verification using Letter Features in Persian Alphabet", Tarbiat Modares University
			\item 2013 Working as co-author and simulator programmer for a Master Thesis about Quantum Key Distribution, Islamic Azad University, Science and Research
		\end{itemize}
	\subsection{Projects}
		\begin{itemize}
			\item InfoSim: A Java based simulation framework which is written for simulating Information Theory problems and scenarios. It allows you to create Channels, Codes, Modulators, Multiplexors all along with telecommunication standards and protocols. It also has a rich and powerful mathematical framework containing packages related to Abstract Algebra, Linear Algebra and Probablities.
		\end{itemize}
		
\section{Professional Open Source Developer}
		\begin{itemize}
			\item Sourceforge developer at \texttt{https://sourceforge.net/u/kamranamini/profile/}
			\item GitHub developer at \texttt{https://github.com/kamcpp}
			\item Open Source development tools: SVN, Git, Maven, Gradle, etc
		\end{itemize}
		
\section{Activities as Instructor}
	\subsection{Mojtama E Fanni E Tehran,MFT}
		\begin{itemize}
			\item Mojtama E Fanni E Tehran,MFT(2011 to present) at Tehran, Iran
			\item Algorithms, Object Oriented Paradigm, OOP in C++, C\#.NET
			\item C++ in Linux and Windows
			\item Java fundamental and Java advanced topics
			\item JEE architecture and technologies: CDI, Servlets, JDBC, EJB, JCA, JPA, JAX-WS, JAX-WS, etc.
			\item Web Services (SOAP and RESTful)
			\item JDBC, JPA and Hibernate, TopLinks.
			\item JVM Technologies: Groovy, Scala, Naashorn Engine
			\item Linux Essentials
			\item Development in Linux
			\item Microsoft C\#.NET(.NET Framework Foundation and Framework Design)
			\item ASP.NET WebForms and MVC
			\item Microsoft SQL Server 2008 and 2012(Design, Maintenance and Administration)
			\item MySQL and PostgreSQL DBMS
			\item CIW Web Foundations (Network, Design and Develop)
			\item Web development with PHP 
		\end{itemize}
	\subsection{Novin Parsian Master Educational Institute}
		\begin{itemize}
			\item Novin Parsian Master Educational Institute(2009 to 2012) at Tehran, Iran
			\item Microsoft C\#.NET Programming
			\item Microsoft SQL Server 2005, 2008 Design and Administration
			\item Oracle 10g Design and Administration
		\end{itemize}
	\subsection{Padideh Educational Institute}
		\begin{itemize}
			\item Padideh Educational Institute(2006 to 2012) at Tehran, Iran
			\item Software related academic courses e.g Algorithms, Data Structure, Algorithm Design, etc.
			\item Payam Noor University Master Degree, Database Systems(Based on C.J.Date book)
		\end{itemize}
	\subsection{Iran Khodro Co.}
		\begin{itemize}
			\item SeptaNSF Framework as the Framework Specialist and Architect from Septa Co.
		\end{itemize}			
\section{Professional Career as Software Developer/Designer}
    \subsection{Mojtama E Fanni E Tehran, Vanak delegacy}
    	\begin{itemize}
    	    \item Director of Web and Development department
    		\item Working on an educational system called EduSys
    		\item It was Service Oriented. Service layer had been written using Spring WS framework.
    		\item It was Java based and written using Spring frameworks. Client applications were being developed using C\#.NET. They had to use web services to communicate with service layer.
    		\item Another web based interface were also being written. Spring MVC and Spring Security frameworks were used.
    		\item Working with XSD and WSDL files
    		\item Working with SoapUI as an investigator for Soap connections.
			\item PostgreSQL were used as our database management system.
			\item Ubuntu Server distribution were used as our server environment.
    	\end{itemize}
    	
	\subsection{SharifCERT (APA Sharif)}
		 NOTICE : Projects and their details are confidential and I can't mention them here. So, a general overview on used technologies will be given.
		\begin{itemize}
			\item SharifCERT(APA Sharif) (2011 December to 2013) at Tehran, Iran
			\item Network security related applications written using Java, C and C++ under Linux environments
			\item Using SVN and Git alongside Eclipse IDE for Java development
			\item Using Test Driven Development technique for application development
			\item Research and develop about Network Security concerns (L2 and L3)
			\item Working with algorithms and data structures which are involved in security related topics.
			\item Malware analysis, Traffic analysis, Attack analysis
			\item 2012 May, Member of Technical Team, CTF Competitions, Iran's Second National Hacking Contest, Sharif University of Technology. Designer and Auditor for Java Secure Coding Question
			\item 2013 Feb, Member of Technical Team, CTF Competitions, Iran's Second National Hacking Contest, Sharif University of Technology. Designer and Auditor for C++ Secure Coding Question
		\end{itemize}
		
	\subsection{Gostaresh Faryanad Sharif Co. (SEPDCo.)}
		\begin{itemize}
			\item Gostaresh Faryanad Sharif Co.; SEPDCo. (2011 October to 2012 June) at Tehran, Iran
			\item \small \textbf{SEPDCo Gas Network Simulator System} \\
			 A gas network and pipeline designer and simulator application written entirely in C++ using Qt framework. I was developer for the desktop application which was written in C++ and Qt View Framework. I used Qt View Framework classes like QWidget, QMainWindow, QGraphicsScene, QGraphicsView alongside signals and slots with all other needed concepts in Qt. The application had no database and everything was serialized using XML or Binary formatters and written to a file. We were using Microsoft Visual Studio 2008 for developing the project. \\
		\end{itemize}
		
	\subsection{Septa Co.}
		\begin{itemize}
			\item Septa Co. (2009 September-2011 August) at Tehran, Iran
			\item We used Microsoft Visual Studio 2008 as our primary development IDE and Microsoft Visual Source Safe 2005 for our source version control system. Project types were varied from Windows From applications to ASP.NET Web Forms and WCF Services.
			\item \small \textbf{SeptaNSF; A .NET based Framework} \\
			This was a .NET based framework implemented for internal usages and later became a product. It also had a code generator which was able to generate Business Logic Layer entities and their adapters. It was entirely written in C\#.NET 3.5 and I was one of its developers and designers. I was developer and designer for classes related to WCF services, classes related to Inter-Process Communications like Sockets, Pipes and classes related to mathematical concepts related to cryptography. \\
			\item \small \textbf{\texttt{www.1st.ir} B2B Portal} \\
			It was one of the first Iranian Business 2 Business portals. Written in ASP.NET C\#.NET 3.5, using a Microsoft SQL Server 2008 R2 database. All parts relating to WCF web services were developed by me. Also, portal integration with other products using web services were mine. \\
			\item \small \textbf{PayamGostar Application} \\
			A business directory desktop application written entirely in C\#.NET 3.5 and it was using a Microsoft SQL Server 2008 R2 database. It allowed people to search, save and edit their customers and it had a bank of businesses information called business directory. It was integrated with 1st portal (previous section) and bought application instances had to be activated via portal. It was checking user's activation status using WCF web services over a HTTPS channel. In addition, new information for businesses could be synchronized with \texttt{www.1st.ir} portal using WCF web services. \\
			\item \small \textbf{Septa Update Center} \\
			An update/patch system integrated with development procedures to provide updates for company's products. It was entirely written in C\#.NET 3.5. An updater service with a desktop updater application were given to clients for updating their product instances. It was fully designed and implemented by me and now it is working as the main update/patch package for Septa Co. products. It is implemented using WCF technology for downloading patches, updates, checking activation and subscription status and deactivating product instances. \\
			\item \small \textbf{Septa Automation System} \\
			It was an internal ASP.NET C\#.NET 3.5 website for entrance system, salary calculations, product instances, serial code generation, tracking sold products, etc. It was all implemented by me. It was using ASP.NET membership bundle as its authentication system.
		\end{itemize}
	
	\subsection{Shabakeh Pardaz Rayaneh Co.}
		\begin{itemize}
			\item Shabakeh Pardaz Rayaneh Co. (2009 June-2009 September) at Isfahan, Iran
			\item \small \textbf{Isfahan Province Electricity Distribution Center, 121 Call Center} \\
			This was the same project at previous company, Aramin IT, which was transferred to Shabakeh Pardaz Rayaneh Co. and I moved to new company to continue to work on the project. \\
			\item \small \textbf{Recording and Archiving System at Isfahan Central Jails Organization} \\
			The archiving system was implemented using C\#.NET and Microsoft SQL Server 2005. Many database related issues like heavy loads and load balancing, partitioning, indexes, etc were encountered and studied.
		\end{itemize}
		
	\subsection{Aramin IT Co.}
		\begin{itemize}
			\item Aramin IT Co. (2008 March -2009 June) at Isfahan, Iran
			\item \small \textbf{Isfahan Province Electricity Distribution Center, 121 Call Center} \\					
			This was a set of projects using Visual C++ and C\#.NET which they were responsible for handling calls at Isfahan Electricity Distribution Center. There was a service written in C++ which it had to monitor a Panasonic TD 200 call center. The service was using TAPI API for controlling the call center. Also, there were some desktop applications written in C\#.NET which they allowed human agents and managers to monitor, accept, reject and redirect the calls. Database system used for this project was Micrsoft SQL Server 2005.
		\end{itemize}
		
\section{Fields of Expertise}
	\subsection{Source Version Control Systems}
		\begin{itemize}
			\item SVN (Command line and GUI interface)
			\item Git (Command line and GUI interface)
			\item Microsoft Visual Source Safe 2005
			\item Microsoft Team Foundation System
		\end{itemize}
	\subsection{Programming and Scripting Languages}
		\begin{itemize}
			\item \textbf{C, C++}
				\begin{itemize}
					\item I've started using C++ since high school in a 2D Soccer Simulation Robocup team.
					\item I've worked with \texttt{gcc} and \texttt{g++} in Unix-like operating systems like Linux Fedora, Linux Ubuntu and FreeBSD.
					\item I've written some projects with C++ using MS Visual Studio 6.0 and MS Visual Studio 2008 in Windows environments. I've used C++ CLI too but I prefer standard C++.
					\item I've worked with \texttt{boost} and \texttt{Qt} frameworks alongside \texttt{STL} in various projects.
					\item Also I have done some windows desktop applications using Visual C++ 6.0 and above.
					\item C++ secure coding according to CERT C++ Secure Coding.
					\item I've taught C++ programming for more than 4 years
				\end{itemize}
			\item \textbf{Java}
				\begin{itemize}
					\item It is since 2011 which I've started Java seriously as a major part of my career. I was familiar with Java since high school but the professional career has been started since I joined SharifCERT.
					\item Most of my projects were written using J2SE framework and with help of other frameworks like Apache Common, Log4J, etc. I've recently gained valuable experiences about J2EE development. I've been using JBoss and Tomcat as my J2EE Application Servers. I've also implemented applications using HornetQ as their underlying messaging system.
					\item Currently, studying and using Spring framework and its Web MVC framework as my main method for web development using Java. Also, as a security solution I'm studying and using Spring Security framework for securing my web and desktop applications.
					\item I really like the idea of \texttt{Dependency Injection} and I use it a lot while I'm designing an application. It has influenced my way of thinking and I use it even I'm using other languages like C++ for development.
					\item Recently I'm using Test Driven Development paradigm. I've found it very useful while using Agile methodology. I use JUnit framework for implementing test units.
					\item I use Hibernate (JPA annotation based and XML based configurations) as my persistence framework.
					\item I use Maven build system as my primary solution for compiling and publishing java applications. I've also worked with Ant build system. I've used even \texttt{Makefiles} to compile Java applications.
					\item I use Apache Tiles framework as my View framework while using Spring or other Web MVC frameworks.
					\item A bit experience with Seam framework.
					\item I've used \texttt{jnetpcap} library as the only available Java solution for capturing and analyzing individual network packets.
					\item I've used Java NIO as a solution for implementing massive IO operations. I've used it to open multiple TCP and UDP ports in not blocking mode.
					\item Most of my projects have been written in Ubuntu and Mint using Eclipse IDE (Kepler edition).
					\item Most of the projects were server side and they had to be written very carefully. They had to work under heavy loads, they had to respond to many requests in a little fraction of time and most of them were 24/7 services.
					\item I've worked with \texttt{java} and \texttt{javac} as Java command line launcher and compiler.
					\item Java secure coding according to The CERT Oracle Secure Coding for Java is also part of my career.
					\item I've used Spring-WS for creating web services.
				\end{itemize}
			\item \textbf{C\#.NET}
				\begin{itemize}
					\item I've written several applications using C\#.NET 3.5 both Windows Form and Web applications.
					\item I've developed various projects with ASP.NET Web Forms.
					\item I've written some projects using C\#.NET 4 but I'm more experienced in C\#.NET 3.5.
					\item I've taught C\#.NET for 2 years based on MCTS Standard.
				\end{itemize}
			\item \textbf{Perl} 
				\begin{itemize}
					\item It is since 2011 which I have started programming with Perl. I've worked with versions 5.8 and 5.14.
					\item I've used Perl to write an Apache HTTP module. Then it became a web MVC framework called OpenPerl::MVC. It is not completed yet but once it gets ready I'm going to publish it on CPAN archive.
				\end{itemize}
			\item \textbf{PHP}
				\begin{itemize}
					\item I've done some websites using PHP but all of them were for testing purposes. Honestly, this was my first server-side programming language and I changed it to Perl or Java on Linux environments. I've used PHP alongside with MySQL as its database system.
				\end{itemize}
			\item \textbf{Unix/Linux Shell Scripting}
				\begin{itemize}
					\item I do all of my OS related jobs using shell scripts under Linux environments. Installers, Package preparations, Folder layout makings and many other tasks can be done using shell scripts. I use them heavily while I'm working on a project. They are great alongside Perl scripts and give you a lot of power to automate and centralize system routines.
				\end{itemize}
		\end{itemize}
		
	\subsection{Operating Systems}
		\begin{itemize}
			\item \textbf{Ubuntu,Debian}
				\begin{itemize}
					\item Version 10.x and 11.x
					\item Programming with C, C++ and Java
					\item Shell Programming
					\item Socket programming
					\item Kernel module programming
					\item Character devices programming
					\item 24/7 active daemons using Java and C++
				\end{itemize}
			\item \textbf{CentOS,Fedora}
				\begin{itemize}
					\item Version 5.x and 6.x
					\item Programming with C++ and Perl
					\item Shell Programming
					\item Socket programming
					\item Web applications using Perl
				\end{itemize}
			\item \textbf{FreeBSD}
				\begin{itemize}
					\item Version 8.4
					\item Programming with C
					\item Kernel module programming
					\item Character devices programming
					\item Socket programming
				\end{itemize}
			\item \textbf{Microsoft Windows}
				\begin{itemize}
					\item XP, Vista and 7
					\item Programming with C++ and C\#.NET
					\item Socket Programming
				\end{itemize}
		\end{itemize}
		
	\subsection{Database Systems}
		\begin{itemize}
			\item \textbf{PostgreSQL}
				\begin{itemize}
					\item PostgresSQL 8.4 under Ubuntu and Debian environments
					\item Configuring database engine using configuration files like pg\_hba.conf, etc.
					\item Used as database systems with Hibernate framework in Java projects.
					\item Used \texttt{psql} client admin tool alongside shell scripts to automate database installation and import steps.
				\end{itemize}
			\item \textbf{Microsoft SQL Server}
				\begin{itemize}
					\item 2005 and 2008 Versions
					\item Database design and Implementation
					\item I've taught Microsoft SQL Server 2008 Design and Administration for 2 years based on MCTS Standard.
				\end{itemize}
			\item \textbf{MongoDB}
				\begin{itemize}
					\item Version 2.0.5
					\item Recently I've started a project which uses MongoDB as its noSQL and document oriented database management system.
					\item I've used MongoDB C\# wrapper which allows you to use MongoDB in C\#.NET projects. Also have used its C++ driver and wrapper which is MongoDB's first default wrapper.
				\end{itemize}
			\item \textbf{Oracle}
				\begin{itemize}
					\item Oracle 10g and 11g
					\item Database design
					\item A little experience about administration. I've taught some courses related to Oracle Database Design and Administration.
				\end{itemize}
		\end{itemize}
	
	\subsection{APIs and Frameworks}
		\begin{itemize}
			\item \textbf{Telephony API} 
				\begin{itemize}
					\item TAPI 2.0 and TAPI 3.0
					\item Used with C, C++ and C\#.NET
					\item Programming with WAPI (Wave API) alongside TAPI for voice recording and IVR(Interactive Voice Response) systems
					\item Familiar with JTAPI (Java Telephony API)
				\end{itemize}
			\item \textbf{Packet Capturing}
				\begin{itemize}
					\item libpcap with C++ in Ubuntu distribution
					\item JNetPcap, a Java wrapper for libpcap, in Ubuntu distribution
					\item Used in applications for Network Traffic and Packet Flow analysis
				\end{itemize}
			\item \textbf{Qt C++ Framework}
				\begin{itemize}
					\item Used Qt data structural and common classes like \texttt{QList}, \texttt{QMap}, etc
					\item Socket programming
					\item GUI programming, worked with \texttt{QWidget} objects, Signals, Slots, etc
				\end{itemize}
			\item \textbf{Boost C++ Framework}
				\begin{itemize}
					\item Familiar with data structures and common classes
				\end{itemize}
			\item \textbf{O/R Mappers}
				\begin{itemize}
					\item LLBLGen
					\item Microsoft Entity Framework
					\item Hibernate for Java (Used with PostgreSQL and MySQL)
				\end{itemize}
			\item \textbf{Windows Communication Foundation(WCF)}
				\begin{itemize}
					\item Transport and Message security models
					\item IIS Hosting and Windows Service Hosting
					\item Basic and WS bindings, Custom bindings
				\end{itemize}
			\item \textbf{Microsoft Hooking framework}
				\begin{itemize}
					\item Dll injection
					\item Writing hooks for monitoring and redirecting Windows Calls.
				\end{itemize}
		\end{itemize}		
\section{Worked Projects}
			This is the table of projects which I've done or been a part of it in my professional career. They can be defined and started whether by companies or by myself. Each project has an index which can be used for reading below table.
			
			\begingroup
			\footnotesize
			\begin{longtable}{ c c c c c c }
			\caption{Worked Projects}
			\label{table:nolin}
			\tabularnewline
				\hline
				Project Index & Languages & DBMS & Roles & Libraries\\
				\hline
				1 & C++,C\#.NET & SQL Server & Developer, Designer & TAPI 2.0 \\
				2 & C\#.NET & SQL Server & Developer, Designer & - \\
				3 & C\#.NET & - & Developer, Designer \\
				4 & C\#.NET & SQL Server & Developer & WCF Services, ASP.NET \\
				5 & C\#.NET & SQL Server & Developer, Designer & WCF Services, Interop \\
				6 & C\#.NET & SQL Server & Developer, Designer & WCF Services, Compression \\
				7 & C\#.NET & SQL Server & Developer, Designer & ASP.NET \\
				8 & C++ & - & Developer, Designer & Qt Framework \\
				9 & C++ & - & Idea Owner, Developer and Designer & \\
				10 & Perl & - & Idea Owner, Developer and Designer & \\
				11 & C++ & MongoDB & Idea Owner, Developer and Designer & Qt, Boost \\
				12 & Java,C,C++ & PostgreSQL & Developer, Designer & libpcap, J2EE, Qt \\
				\hline
			\end{longtable}
			\endgroup

		 	 1: Isfahan Province Electricity Distribution Center, 121 Call Center \\
			 \indent 2: Recording and Archiving System at Isfahan Central Jails Organization \\ 
			 \indent 3: SeptaNSF; A .NET based Framework \\
			 \indent 4: 1st.ir B2B Portal \\
			 \indent 5: PayamGostar Application \\
			 \indent 6: Septa Update Center \\
			 \indent 7: Septa Automation System \\
			 \indent 8: SEPDCo Gas Network Simulator System \\
			 \indent 9: ImgProcLib; A C++ Image Processing Library \\
			\indent 10: OpenPerl::MVC web framework \\
		    \indent 11: MafiaClub; An online multiplayer game \\
			\indent 12: SharifCERT projects \\
					
\end{document}
